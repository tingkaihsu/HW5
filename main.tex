\documentclass[12pt,a4paper]{article}

% Packages for enhanced functionality
\usepackage{braket}
\usepackage{physics}
\usepackage{graphicx}
\usepackage{times}
\usepackage[export]{adjustbox}
\usepackage{listings}
\usepackage{mathcomp}
\usepackage{hyperref}
\usepackage{bm,amsmath}
\usepackage{float}
\usepackage{indentfirst}
\usepackage{bigints}
\usepackage{color}
\usepackage[margin=2.2cm]{geometry}

% Hyperlink setup
\hypersetup{
    colorlinks=true,
    linkcolor=blue,
    filecolor=magenta,
    urlcolor=cyan,
    pdftitle={HW3: Thank You For Smoking},
    pdfpagemode=FullScreen,
}

% Listings setup for code formatting
\definecolor{dkgreen}{rgb}{0,0.6,0}
\definecolor{gray}{rgb}{0.5,0.5,0.5}
\definecolor{mauve}{rgb}{0.58,0,0.82}
\lstset{
    frame=tb,
    language=Python,
    aboveskip=3mm,
    belowskip=3mm,
    stepnumber=1,
    showstringspaces=false,
    columns=flexible,
    basicstyle={\small\ttfamily},
    numbers=left,
    numberstyle=\color{gray},
    keywordstyle=\color{blue},
    commentstyle=\color{dkgreen},
    stringstyle=\color{mauve},
    breaklines=true,
    breakatwhitespace=true,
    tabsize=3
}

% Number equations by section
\numberwithin{equation}{section}

% Document metadata
\title{HW5: How to Pick a Research Topic}
\author{
    Ting-kai Hsu\\
    B11901097\\
    \textit{Department of Electrical Engineering, NTU}
}
\date{\today}

\begin{document}

% Title page
\maketitle

% Table of contents
\tableofcontents
\section{Introduction}
We overviewed Hawkings' life in searching the theory of unification in the film, \textit{the theory of everything}. In this homework, please read a paper \cite{doi:10.1126/science.177.4047.393} and watch two videos of two Nobel laureates, P.Anderson and S.Weinberg. Hawkings and most of the physicists believe there is a simple theory that is able to describe everything in our universe, and such success have been accomplished partially in last century, \textbf{the Standard Model} can describe three elementary forces in a compact way. Theorists now work hard on combining the forth force, gravity into this framework.

Although Newton's law is capable to describe a falling ball precisely, as more and more conditions are taken into account (like dragging force of air and so on), the simple model can become more and more complicated and unpredictable eventually. This point of view is taken by Anderson, which says in his paper \cite{doi:10.1126/science.177.4047.393} that physics laws hold in few-body system, but fail in many-body system, where there might be interesting phenomena that cannot be predicted by the physics laws.
% References
\bibliographystyle{plain}
\bibliography{ref}

\end{document}
