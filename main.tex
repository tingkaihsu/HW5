\documentclass[a4paper,12pt]{article}
\usepackage{hyperref}

\begin{document}

\section*{Overview}

In the film \textit{The Theory of Everything}, we explored the life of Stephen Hawking and his pursuit of a unified theory of the universe. This homework assignment builds upon that theme by asking you to engage with key works from two Nobel laureates, P. W. Anderson and Steven Weinberg. Specifically, you are required to read Anderson's seminal paper \cite{doi:10.1126/science.177.4047.393} and watch two videos \cite{Weinberg2021, Weinberg2021Part2} featuring Weinberg's perspectives.

Hawking, along with many physicists, believed in the existence of a simple, fundamental theory capable of explaining all phenomena in the universe. Substantial progress toward this goal was achieved in the 20th century through the development of the \textbf{Standard Model}, which provides a compact framework for describing three of the four fundamental forces of nature. The current challenge for theorists lies in integrating the fourth force, gravity, into this framework.

While Newton’s laws effectively describe phenomena such as a falling ball, complications arise as additional factors (e.g., air resistance) are introduced. Such complexities often render simple models increasingly intricate and less predictable. P. W. Anderson’s perspective, articulated in his paper \cite{doi:10.1126/science.177.4047.393}, emphasizes that while physical laws govern small-scale systems, they often fail to capture the emergent phenomena observed in complex, many-body systems.

Conversely, Steven Weinberg advocates for the pursuit of a unified theory, expressing confidence that such a theory will eventually be realized. In his interviews, he discusses reductionism—the idea that understanding a system’s fundamental components leads to a complete understanding of the whole—and contrasts it with holism, a framework asserting that systems must be studied as integrated wholes. These concepts, which span multiple scientific disciplines, represent distinct approaches to scientific inquiry \cite{WikipediaReductionism, WikipediaHolism}.

\section*{Requirements}

You are tasked with carefully reviewing Anderson's paper and Weinberg’s videos, analyzing the strengths and weaknesses of their respective research perspectives. Based on your analysis, decide which approach you would adopt if you were conducting research, providing clear and reasoned arguments. Personal opinions without substantiation should be avoided.

\section*{Materials}

\begin{itemize}
    \item \textbf{Film}: \textit{The Theory of Everything}
    \item \textbf{Reading Assignment}: Anderson, P. W. (1972). \textit{More is Different} \cite{doi:10.1126/science.177.4047.393}
    \item \textbf{Videos}: Two 17-minute interviews with Steven Weinberg \cite{Weinberg2021, Weinberg2021Part2}
\end{itemize}

\begin{thebibliography}{99}
    \bibitem{doi:10.1126/science.177.4047.393} P. W. Anderson, ``More is Different,'' \textit{Science}, vol. 177, no. 4047, pp. 393-396, 1972. DOI: \url{https://doi.org/10.1126/science.177.4047.393}.
    \bibitem{Weinberg2021} S. Weinberg, ``Interview Part 1,'' 2021. \url{https://example.com/weinberg1}.
    \bibitem{Weinberg2021Part2} S. Weinberg, ``Interview Part 2,'' 2021. \url{https://example.com/weinberg2}.
    \bibitem{WikipediaReductionism} Wikipedia contributors, ``Reductionism,'' \textit{Wikipedia}. \url{https://en.wikipedia.org/wiki/Reductionism}.
    \bibitem{WikipediaHolism} Wikipedia contributors, ``Holism,'' \textit{Wikipedia}. \url{https://en.wikipedia.org/wiki/Holism}.
\end{thebibliography}

\end{document}
